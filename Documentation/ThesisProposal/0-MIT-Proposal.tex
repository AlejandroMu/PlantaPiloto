\documentclass[letterpaper,11pt]{report}

\textheight=22cm \textwidth=16.5cm \hoffset=-2cm \voffset=-1cm

\usepackage{inputenc} % set input encoding (not needed with XeLaTeX)

\usepackage[font=small,format=plain,labelfont=bf,up,textfont=it,up]{caption} %CAPTION for the figures
\usepackage{multirow} %multiples lineas en las tablas
\usepackage{graphicx} % support the \includegraphics command and options
\usepackage{array} % for better arrays (eg matrices) in maths
\usepackage{verbatim} % adds environment for commenting out blocks of text & for better verbatim
\usepackage{framed}
\usepackage[plainpages=false, colorlinks=false, pdfborder={0 0 0}]{hyperref}
%\usepackage{paralist} % very flexible & customisable lists (eg. enumerate/itemize, etc.)
%\usepackage{amsmath}
%\usepackage{subfig} % make it possible to include more than one captioned figure/table in a single float
%\usepackage{booktabs} % for much better looking tables

%Configuracion del estilo de la página
\usepackage{fancyhdr} % This should be set AFTER setting up the page geometry
\pagestyle{fancy} % options: empty , plain , fancy
\renewcommand{\headrulewidth}{0pt} % Configura la linea del encabezado
\lhead{}\chead{}\rhead{} %Configura el encabezado
\lfoot{}\cfoot{\thepage}\rfoot{} %Configura el pie de pagina


\newenvironment{subgroup}{$\left\{\tabular{l}} {\endtabular\right.$}

\usepackage{sectsty}
\allsectionsfont{\sffamily\mdseries\upshape} % (See the fntguide.pdf for font help)

\newcommand{\etal}[0]{\textit{et al.}} % required packages


% ==========================================================================
\begin{document}


% Title information
\begin{titlepage}
\begin{center}

% Upper part of the page
~\\

% Title
{ \Huge \textbf{The Subject-Matter for}\\[0.2cm]
\Huge \textbf{the Kind of Software Systems of Your Interest}}\\[3cm]


\textsc{\Large Master Thesis Proposal}\\ [1cm]

Presented in partial fulfillment to obtain the Title of \\
Magister in Informatics and Telecommunications \\[3cm]

by \\[0.5cm]

\textsc{\Large Alejandro Muñoz }\\[1cm]

\large{Advisor:} Gabriel Tamura \\

\vfill

{\Large Department of Information and Communication Technologies}\\[0.3cm]
{\Large Faculty of Engineering}\\[0.5cm]

\includegraphics[width=0.22\textwidth]{logos/logo-icesi.png}\\[0.3cm]

\large 2012\\

\end{center}

\end{titlepage}


\pagenumbering{roman}

%Tablas de contenido
\tableofcontents
\listoftables
\listoffigures
%\pagebreak

%Sections
\sloppy

\abstract{Summary of your Thesis Proposal. In English and then in Spanish.}

\pagenumbering{arabic}

% =========================================================================
\chapter{Context and Motivation}
\label{cha:MotivationBackground}


% =========================================================================
\chapter{Problem Definition}
\label{cha:ProblemDef}

As mentioned in Chapter \ref{cha:MotivationBackground}, the separation of concerns is crucial to facilitate the adaptation as a software system independently the adaptation mechanism. To understand the relevance of this proposal it is important to define the main concepts involved in this project: Autonomic computing, Self-Adaptive Systems, Component-Based Software, Software Architecture, Reference Models and Reference Architectures \cite{delemos-et-al:2011:software-engineering-for-sas-roadmap-2}.


\section{Problem Statement}
\label{sec:ProblStatmt}

State your addressed problem as clearly and concisely as possible.

Never mix the problem with the solution.


% ----------------------------------------------------------------------------
\section{Challenges (or Hypotheses)}
\label{sec:Challenges}




% =========================================================================
\chapter{Objectives}
\label{cha:Objectives}

\section{General}

\section{Specific}



% =========================================================================
\chapter{Theoretical Background -- Overview}
\label{cha:TheoreticalBackground}


% =========================================================================
\chapter{Preliminary State of the Art}
\label{cha:StateOfTheArt}



% =========================================================================
\chapter{Methodology}
\label{cha:Methodology}

Explain the steps required to achieve the stated objectives. Support them as required with appropriate methods \cite{creswell:2009:scientific-research-book}.
Refine the  specific objectives and, more importantly, explain how are you going to validate your proposed solution.


% =========================================================================
\chapter{Expected Results}
\label{cha:ExpectedResults}


% =========================================================================
\chapter{Tentative Project Schedule}
\label{cha:ProjectSchedule}




%%%%%%%%%%%%%%
% BibTeX users 
%\bibliographystyle{bib/splncs}
\bibliographystyle{alphaabbr}
\bibliography{self-adaptation}

%%%%%%%%%%%%%%
\end{document}